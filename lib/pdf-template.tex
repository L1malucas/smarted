\documentclass[a4paper,12pt]{article}
\usepackage[utf8]{inputenc}
\usepackage[T1]{fontenc}
\usepackage[portuguese]{babel}
\usepackage{amsmath}
\usepackage{geometry}
\usepackage{booktabs}
\usepackage{graphicx}
\usepackage{longtable}
\usepackage{xcolor}
\usepackage{hyperref}
\usepackage{fancyhdr}
\usepackage{colortbl}
\usepackage{array}
\usepackage{tabularx}

\geometry{margin=0.8in}

% Definir cores
\definecolor{tealheader}{RGB}{0,102,102}
\definecolor{yellowaccent}{RGB}{255,215,0}
\definecolor{lightgray}{RGB}{240,240,240}

% Configurar cabeçalho
\pagestyle{fancy}
\fancyhf{}
\fancyhead[L]{
    \colorbox{tealheader}{
        \parbox{12cm}{
            \textcolor{white}{\textbf{Empresa}}\\
            \textcolor{white}{Endereço, Cidade, Estado e CEP}\\
            \textcolor{white}{Telefone Telefone Fax Fax}
        }
    }
}
\fancyhead[R]{
    \colorbox{yellowaccent}{
        \parbox{3cm}{
            \centering
            \textbf{Nome do}\\
            \textbf{logotipo}
        }
    }
}

% Configurar rodapé
\fancyfoot[C]{
    \hrule
    \vspace{2mm}
    \small
    \textbf{© 2025 - Todos os direitos reservados}\\
    Este documento é confidencial e propriedade da empresa. Reprodução ou distribuição não autorizada é proibida.\\
    Gerado em: \today
}

% Redefinir altura do rodapé
\setlength{\footskip}{30pt}

\renewcommand{\headrulewidth}{0pt}
\renewcommand{\footrulewidth}{0.4pt}

\begin{document}

% Título principal
\vspace{1cm}
\begin{center}
    {\Huge \textbf{\textcolor{tealheader}{RELATÓRIO DE STATUS DO PROJETO}}}
\end{center}
\vspace{1cm}

% Seção: Resumo do Projeto
\section*{\textcolor{tealheader}{RESUMO DO PROJETO}}

\begin{tabularx}{\textwidth}{|>{\columncolor{tealheader}\color{white}\bfseries}p{3cm}|>{\columncolor{lightgray}}p{4cm}|>{\columncolor{tealheader}\color{white}\bfseries}p{3cm}|>{\columncolor{lightgray}}X|}
\hline
DATA DO RELATÓRIO & \VAR{reportDate} & PREPARADO POR & \VAR{preparedBy} \\
\hline
NOME DO PROJETO & \multicolumn{3}{p{11cm}|}{{\VAR{projectName}}} \\
\hline
\end{tabularx}

\vspace{0.5cm}

% Seção: Relatório do Status
\section*{\textcolor{tealheader}{RELATÓRIO DO STATUS}}

\colorbox{yellowaccent}{
    \parbox{\textwidth}{
        \textbf{Para começar agora, toque em qualquer espaço reservado para texto (como este) e comece a digitar para substituir por próprio seu texto.}
    }
}

\vspace{0.5cm}

% Seção: Visão Geral do Projeto
\section*{\textcolor{tealheader}{VISÃO GERAL DO PROJETO}}

\begin{tabularx}{\textwidth}{|>{\columncolor{tealheader}\color{white}\bfseries}p{2.5cm}|>{\columncolor{tealheader}\color{white}\bfseries}p{2.5cm}|>{\columncolor{tealheader}\color{white}\bfseries}p{2.5cm}|>{\columncolor{tealheader}\color{white}\bfseries}p{2cm}|>{\columncolor{tealheader}\color{white}\bfseries}X|}
\hline
TAREFA & \% CONCLUÍDA & DATA DE ENTREGA & DRIVER & ANOTAÇÕES \\
\hline
\rowcolor{lightgray}
Vagas Criadas & \VAR{metrics.vagasCriadas}\% & \VAR{metrics.dataEntregaVagas} & \VAR{metrics.responsavelVagas} & \VAR{metrics.anotacoesVagas} \\
\hline
\rowcolor{white}
Candidatos Cadastrados & \VAR{metrics.candidatosCadastrados}\% & \VAR{metrics.dataEntregaCandidatos} & \VAR{metrics.responsavelCandidatos} & \VAR{metrics.anotacoesCandidatos} \\
\hline
\rowcolor{lightgray}
Contatos Realizados & \VAR{metrics.contatosRealizados}\% & \VAR{metrics.dataEntregaContatos} & \VAR{metrics.responsavelContatos} & \VAR{metrics.anotacoesContatos} \\
\hline
\rowcolor{white}
Matches Gerados & \VAR{metrics.matches}\% & \VAR{metrics.dataEntregaMatches} & \VAR{metrics.responsavelMatches} & \VAR{metrics.anotacoesMatches} \\
\hline
\end{tabularx}

\vspace{0.5cm}

% Seção: Visão Geral do Orçamento
\section*{\textcolor{tealheader}{VISÃO GERAL DO ORÇAMENTO}}

\begin{tabularx}{\textwidth}{|>{\columncolor{tealheader}\color{white}\bfseries}p{3cm}|>{\columncolor{tealheader}\color{white}\bfseries}p{2.5cm}|>{\columncolor{tealheader}\color{white}\bfseries}p{2.5cm}|>{\columncolor{tealheader}\color{white}\bfseries}p{2.5cm}|>{\columncolor{tealheader}\color{white}\bfseries}X|}
\hline
CATEGORIA & GASTO & \% DO TOTAL & NO ORÇAMENTO? & ANOTAÇÕES \\
\hline
\rowcolor{lightgray}
Recrutamento & R\$ \VAR{budget.recrutamento} & \VAR{budget.percRecrutamento}\% & \VAR{budget.statusRecrutamento} & \VAR{budget.anotacoesRecrutamento} \\
\hline
\rowcolor{white}
Tecnologia & R\$ \VAR{budget.tecnologia} & \VAR{budget.percTecnologia}\% & \VAR{budget.statusTecnologia} & \VAR{budget.anotacoesTecnologia} \\
\hline
\rowcolor{lightgray}
Marketing & R\$ \VAR{budget.marketing} & \VAR{budget.percMarketing}\% & \VAR{budget.statusMarketing} & \VAR{budget.anotacoesMarketing} \\
\hline
\end{tabularx}

\vspace{0.5cm}

% Seção: Risco e Histórico de Problema
\section*{\textcolor{tealheader}{RISCO E HISTÓRICO DE PROBLEMA}}

\begin{tabularx}{\textwidth}{|>{\columncolor{tealheader}\color{white}\bfseries}p{6cm}|>{\columncolor{tealheader}\color{white}\bfseries}p{4cm}|>{\columncolor{tealheader}\color{white}\bfseries}X|}
\hline
PROBLEMA & ATRIBUÍDO A & DATA \\
\hline
\BLOCK{ for problem in problems }
\rowcolor{lightgray}
\VAR{problem.descricao} & \VAR{problem.responsavel} & \VAR{problem.data} \\
\hline
\BLOCK{ endfor }
\end{tabularx}

\vspace{0.5cm}

% Seção: Atividade do Usuário
\section*{\textcolor{tealheader}{ATIVIDADE DO USUÁRIO}}

\begin{longtable}{|>{\columncolor{tealheader}\color{white}\bfseries}p{4cm}|>{\columncolor{tealheader}\color{white}\bfseries}p{3cm}|>{\columncolor{tealheader}\color{white}\bfseries}p{3cm}|}
\hline
DIA & LOGINS & AÇÕES \\
\hline
\endfirsthead
\hline
\rowcolor{tealheader}
\textcolor{white}{\textbf{DIA}} & \textcolor{white}{\textbf{LOGINS}} & \textcolor{white}{\textbf{AÇÕES}} \\
\hline
\endhead
\BLOCK{ for item in userActivity }
\ifodd\value{rownum}
\rowcolor{lightgray}
\fi
\VAR{item.name} & \VAR{item.logins} & \VAR{item.acoes} \\
\hline
\BLOCK{ endfor }
\end{longtable}

\vspace{0.5cm}

% Seção: Conclusões/Recomendações
\section*{\textcolor{tealheader}{CONCLUSÕES/RECOMENDAÇÕES}}

\colorbox{yellowaccent}{
    \parbox{\textwidth}{
        \textbf{Você acha que um documento apresentado na parede deve ser difícil de formatá-lo? Achou errado! Para aplicar facilmente qualquer formatação, destaque o texto desejado e use as opções de texto na guia Página Inicial. Você pode até mesmo alterar as opções.}
    }
}

\end{document}